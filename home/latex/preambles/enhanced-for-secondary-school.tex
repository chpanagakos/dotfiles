% !TEX program = lualatex
\documentclass[a4paper,11pt]{article}

\usepackage{fontspec}
\usepackage{babel}
\babelprovide[main, import]{greek}
\babelprovide[import]{english}

\babelfont{rm}{Liberation Serif}
\babelfont{sf}{Liberation Sans}
\babelfont{tt}{Liberation Mono}

% math engine
\usepackage{mathtools} % load BEFORE unicode-math
\usepackage{unicode-math}
\setmathfont{TeX Gyre Termes Math}

% ===== DELIMITERS =====
\DeclarePairedDelimiter{\abs}{\lvert}{\rvert}        % |x|
\DeclarePairedDelimiter{\norm}{\lVert}{\rVert}        % ||v||
\DeclarePairedDelimiter{\pqty}{\lparen}{\rparen}      % (x)
\DeclarePairedDelimiter{\angles}{\langle}{\rangle}    % <ψ>

% ===== DIFFERENTIALS & DERIVATIVES =====
\newcommand{\dd}{\symup{d}}                           % upright differential d
\newcommand{\dv}[2]{\frac{\dd #1}{\dd #2}}            % dv/dt
\newcommand{\pdv}[2]{\frac{\partial #1}{\partial #2}} % ∂u/∂x
\newcommand{\dvn}[3]{\frac{\dd^{#3} #1}{\dd #2^{#3}}} % d^n x/d t^n
\newcommand{\pdvn}[3]{\frac{\partial^{#3} #1}{\partial #2^{#3}}} % ∂^n u/∂x^n

% ===== VECTORS & OPERATORS =====
\newcommand{\vect}[1]{\symbf{#1}}                     % Bold vectors
\newcommand{\uvec}[1]{\hat{\symbf{#1}}}               % Unit vectors (î, ĵ, k̂)
\newcommand{\grad}{\symbf{\nabla}}                    % ∇
\newcommand{\divergence}{\symbf{\nabla}\cdot}         % ∇·
\newcommand{\curl}{\symbf{\nabla}\times}              % ∇×

% ===== GREEK TRIG OPERATORS =====
\DeclareMathOperator{\hm}{ημ}  % sin
\DeclareMathOperator{\syn}{συν} % cos
\DeclareMathOperator{\ef}{εφ}   % tan
\DeclareMathOperator{\sef}{σφ}  % cot

% ===== siunitx =====
\usepackage{siunitx}
\sisetup{
  per-mode=symbol,
  output-decimal-marker = {,},
  separate-uncertainty = true,
}

% ===== GREEK ENUMERATION α), β) … =====
\usepackage{enumitem}
\makeatletter
\newcommand{\@greekalph}[1]{%
  \ifcase#1\or α\or β\or γ\or δ\or ε\or ζ\or η\or θ\or ι\or κ\or λ\or μ%
  \or ν\or ξ\or ο\or π\or ρ\or σ\or τ\or υ\or φ\or χ\or ψ\or ω\else\@ctrerr\fi}
\newcommand{\greekalph}[1]{\expandafter\@greekalph\csname c@#1\endcsname}
\AddEnumerateCounter{\greekalph}{\@greekalph}{ω}
\makeatother

% ===== tcolorbox =====
\usepackage[most]{tcolorbox}
\tcbset{colback=white,colframe=black,arc=2pt,boxrule=.5pt}

\newtcolorbox{theory}{title={Θεωρία},fonttitle=\bfseries}
\newtcolorbox{examplebox}{title={Παράδειγμα},fonttitle=\bfseries}
\newtcolorbox{exercisebox}{title={Άσκηση},fonttitle=\bfseries}
\newtcolorbox{summarybox}{title={Συμπέρασμα},fonttitle=\bfseries,colback=black!5}

% =======================
% BEGIN DOCUMENT
% =======================
\begin{document}
\section*{ΠΑΡΑΔΕΙΓΜΑΤΑ ΓΙΑ ΟΛΑ ΤΑ NEWCOMMANDS}

% ---- Paired delimiters ----
\begin{examplebox}
\textbf{Paired delimiters}

\[
\abs{-5}=5,\qquad
\norm{\vect{v}} = \norm*{\begin{bmatrix}3\\4\end{bmatrix}} = 5,
\qquad
\pqty{a+b}^2 = a^2 + 2ab + b^2,
\qquad
\angles{\psi} \text{ χρησιμοποιείται σε Κβαντική.}
\]
\end{examplebox}

% ---- Derivatives ----
\begin{examplebox}
\textbf{Derivatives + Differentials}

\[
\dv{x}{t}=3t^2,\qquad
\pdv{u}{x}=2x+3y,
\]

Higher order:
\[
\dvn{x}{t}{2}=\dv{}{t}\pqty{\dv{x}{t}},\qquad
\pdvn{u}{x}{2}=\frac{\partial^2 u}{\partial x^2}.
\]

Integration w/ upright differential:
\[
s=\int v \,\dd t.
\]
\end{examplebox}

% ---- Vectors & Field Operators ----
\begin{examplebox}
\textbf{Vectors & Operators}

\[
\vect{F} = m\vect{a},\qquad
\uvec{i},\uvec{j},\uvec{k},
\]

\[
\grad \phi,\qquad
\divergence \vect{E} = \frac{\rho}{\varepsilon_0},\qquad
\curl \vect{B} = \mu_0\vect{J}.
\]
\end{examplebox}

% ---- Trigonometry ----
\begin{examplebox}
\textbf{Trig operators (upright Greek)}

\[
\hm 30^\circ=\frac12,\qquad
\syn 60^\circ=\frac{\sqrt3}{2},\qquad
\ef 45^\circ=1.
\]

Ταυτότητα:
\[
\hm^2\theta+\syn^2\theta=1.
\]
\end{examplebox}

% ---- Units ----
\begin{examplebox}
\textbf{SI Units (siunitx)}

\[
g=\SI{9.81}{\meter\per\second\squared},\qquad
F=\SI{12}{\newton},\qquad
P=\SI{50}{\watt},\qquad
\num{6.022e23}
\]
\end{examplebox}

% ---- Enumeration α), β), γ)... ----
\begin{examplebox}
\textbf{Greek enumeration}

\begin{enumerate}[label=\greekalph*)]
\item Ταχύτητα ομοιόμορφης κίνησης
\item Νόμος Νεύτωνα
\item Βασικές Τριγωνομετρικές Ταυτότητες
\end{enumerate}
\end{examplebox}

\begin{summarybox}
Όλα τα \verb|\newcommand| τώρα έχουν πλήρη παραδείγματα.
\end{summarybox}

\end{document}

